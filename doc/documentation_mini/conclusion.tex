% vim.ft=tex
\chapter*{Conclusions}
\addcontentsline{toc}{chapter}{Conclusion}

The goal of this thesis was to design and build a cost-effective two-axis rotary platform for a radar system.
Due to the radar's relatively low resolution and polling rate, the platform could be constructed using standard off-the-shelf components such as bipolar stepper motors and an ESP32 microcontroller.
This approach significantly reduced costs compared to professionally manufactured solutions currently available on the market.

The platform's software was developed with a focus on maximizing performance while ensuring high autonomy in operation.
This design philosophy facilitates seamless integration with other projects, eliminating the need for users to understand the intricacies of system operations.
Movement patterns are simply defined using standard G-code commands inside custom programs and send to the platform which handles their execution autonomously.

Performance measurements demonstrated the platform’s precise speed control, achieving an relative error in the order of $10^{-3}$ \%.
Furthermore, when supplied with commands in advance, the platform transitions between commands with imperceptible pauses, even at high speeds leading to smooth and continuous movement.
Coupled with the frequent uplink of the platform's current position, this capability allows users to make all necessary movement corrections during post-processing of radar data.

By combining affordability, precision, and ease of use, the designed platform serves as a practical and efficient solution for low-bandwith radar-based applications, offering a valuable tool for further research and development.
