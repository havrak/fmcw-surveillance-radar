%%% Please fill in basic information on your thesis, which will be automatically
%%% inserted at the right places. You need to replace ... by real data.

% Type of your thesis:
%	"bc" for Bachelor's
%	"mgr" for Master's
%	"phd" for PhD
%	"rig" for rigorosum
% "sem" for semestral
\def\ThesisType{bc}


\def\ThesisTitle{Two Axis Rotary Platform for Radar System}

\def\ThesisTitleShort{Radar Positioning platform}

\def\ThesisAuthor{Havránek Kryštof}

\def\MontSubmitted{November}

\def\YearSubmitted{2024}

\def\Institution{Czech Technical University in Prague}

\def\Faculty{Faculty of Electrical Engineering}

\def\DepartmentType{Department}

\def\Department{Department of Electromagnetic Field}

\def\Supervisor{Ing. Viktor Adler, Ph.D}

\def\SupervisorsDepartment{Department of Electromagnetic Field}

\def\StudyProgramme{Elektronika a komunikace}

\def\Dedication{%
Dedication.
}

\def\Abstract{%
Following thesis is concerned with designing a two-axis rotary platform for radar-based applications.
Commercial solutions are often prohibitively expensive, overly precise, and lack frequent positional updates essential for radar data post-processing. To address this, a cost-effective platform was developed using off-the-shelf components.
Control of the platform employs a three-layer command processing architecture, enabling seamless transitions between successive commands and timely, deterministic movement.
Performance testing demonstrated a relative error of $10^{-3}$ \% in speed control and imperceptible delays during command switching.
The resulting platform offers reliable and efficient operation, tailored to radar needs, providing a practical, affordable alternative to existing commercial systems.
}

% 3 to 5 keywords (recommended) separated by \sep
% Keywords are useful for indexing and searching for the theses by topic.
\def\ThesisKeywords{%
Radar platform\sep Two Axis Platform \sep Positioning Plaform \sep ESP32 \sep G-code Intepreter \sep Mechatronics
}

% If any of your metadata strings contains TeX macros, you need to provide
% a plain-text version for use in XMP metadata embedded in the output PDF file.
% If you are not sure, check the generated thesis.xmpdata file.
\def\AuthorXMP{\ThesisAuthor}
\def\TitleXMP{\ThesisTitle}
\def\KeywordsXMP{\ThesisKeywords}
\def\AbstractXMP{\Abstract}

% If your abstracts are long and do not fit in the infopage, you can make the
% fonts a bit smaller by this setting. (Also, you should try to compress your abstract more.)
\def\InfoPageFont{}
%\def\InfoPageFont{\small}  % uncomment to decrease font size

% If you are studing in a Czech programme, you also need to provide metadata in Czech:
% (in English programmes, this is not used anywhere)

\def\ThesisTitleCS{Dvojosá rotační plaforma pro radar}
\def\DepartmentCS{Katedra elektromagnetického pole}
\def\DepartmentTypeCS{Katedra}
\def\SupervisorsDepartmentCS{Katedra elektromagnetického pole}
\def\StudyProgrammeCS{Elektronika a komunikace}

\def\ThesisKeywordsCS{%
Radarová platforma\sep Dvojosá polohovací platforma \sep EPS32 \sep G-code \sep Mechatronika
}

\def\AbstractCS{%
Práce se zabývá návrhem dvouosé rotační platformy pro radarové aplikace. Komerční řešení jsou často nepřiměřeně drahá, nadměrně přesná a postrádají časté aktualizace polohy, které jsou nezbytné pro následné zpracování radarových dat. Pro řešení tohoto problému byla navržena cenově dostupná platforma využívající běžně dostupné komponenty. Řízení platformy využívá tří stupňovou architekturu zpracování příkazů, která umožňuje plynulé přechody mezi za sebou jdoucími příkazy a deterministický pohyb. Testování  prokázalo relativní chybu $10^{-3}$ \% v rychlosti a nepostřehnutelné zpoždění při přepínání příkazů. Výsledná platforma nabízí spolehlivé a efektivní řízení přizpůsobené potřebám radaru. Představuje tak praktickou a cenově dostupnou alternativu ke stávajícím komerčním systémům.
}

