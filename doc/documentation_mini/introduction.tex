
\chapter*{Introduction}
\addcontentsline{toc}{chapter}{Introduction}

While two-axis rotary platforms are fairly common in the world of machining, they are often prohibitively expensive, and their operational parameters are frequently unnecessarily precise \cite{carl}.
As for an application as radar platform where the angular width of the main lobe can span several degrees \cite{sidarMAN} high precision is not required.
Additionally, most driving boards, typically in the form of G-code/M-code interpreters, are not designed to provide frequent uplink about their current position \cite{duet}.
Feature that is crucial in order to be able to counteract the movement of the platform in post processing.

This thesis focuses on designing and building a cost-effective two-axis rotary platform with parameters purposefully chosen to match the requirements of a radar system.
This necessitates achieving timely and deterministic movement of the stepper motors, allowing radar data processing to accurately account for platform motion during post-processing.
In addition the platform needs to be as easy to operate for an end user as possible.
Allowing easy integration into projects.

To achieve this, a three-layer command processing architecture is proposed and implemented.
This architecture enables almost seamless transitions between commands.
Additionally, this approach allows for preprogramming entire movement sequences that can be executed continuously without any user intervention.

The first chapter  outlines the basic design requirements for the platform based on radar parameters.
The following chapter focuses on the hardware aspects of the project, including the electronic components and mechanical design.
The final chapter is dedicated to the software that controls the platform.
