%%% Please fill in basic information on your thesis, which will be automatically
%%% inserted at the right places. You need to replace ... by real data.

% Type of your thesis:
%	"bc" for Bachelor's
%	"mgr" for Master's
%	"phd" for PhD
%	"rig" for rigorosum
% "sem" for semestral
\def\ThesisType{bc}
\newcommand{\sidar}{SiRad Easy\textsuperscript{\copyright} }


\def\ThesisTitle{Surveillance FMCW Radar}

\def\ThesisTitleShort{Surveillance FMCW Radar}

\def\ThesisAuthor{Havránek Kryštof}

\def\MontSubmitted{May}

\def\YearSubmitted{2024}

\def\Institution{Czech Technical University in Prague}

\def\Faculty{Faculty of Electrical Engineering}

\def\DepartmentType{Department}

\def\Department{Department of Electromagnetic Field}

\def\Supervisor{Ing. Viktor Adler, Ph.D}

\def\SupervisorsDepartment{Department of Electromagnetic Field}

\def\StudyProgramme{Elektronika a komunikace}

\def\Dedication{%
	I would like to express my gratitude to my supervisor, Ing. Viktor Adler, Ph.D., for his guidance and support throughout the process of writing this thesis.
	Their expertise greatly contributed with me understading this field and had a significant impact on the final outcome of my work.
}

\def\Abstract{%
	This thesis concerns itself with realization of general purpose surveillance radar based on SiRad Easy FMCW module, allowing range measurements of targets with low power consumption.
  To achieve this, the capabilities of the module are first analyzed to identify its limitations.
	Subsequently, a custom two-axis rotary platform is designed and constructed to provide mechanical beam steering, facilitating 3D scanning.
	The platform is driven by an ESP32C6 microcontroller and employs G-code-like commands as its API.
	Finally, MATLAB is used to integrate the platform with the radar module and to process the collected data.
	Although the radar's performance limitations prevent tracking of moving targets, the system is capable of monitoring a wide variety of environments.
	Be it close-range applications requiring high resolution or long-range monitoring of targets up to 300 meters.
	For this purpose, the processing pipeline is designed to be versatile, allowing for a wide range of configurations.
}

% 3 to 5 keywords (recommended) separated by \sep
% Keywords are useful for indexing and searching for the theses by topic.
\def\ThesisKeywords{%
	FMCW, Surveillance Radar, ESP32, MATLAB, Digital Signal Processing, 3D Printing, G-code
}

% If any of your metadata strings contains TeX macros, you need to provide
% a plain-text version for use in XMP metadata embedded in the output PDF file.
% If you are not sure, check the generated thesis.xmpdata file.
\def\AuthorXMP{\ThesisAuthor}
\def\TitleXMP{\ThesisTitle}
\def\KeywordsXMP{\ThesisKeywords}
\def\AbstractXMP{\Abstract}

% If your abstracts are long and do not fit in the infopage, you can make the
% fonts a bit smaller by this setting. (Also, you should try to compress your abstract more.)
\def\InfoPageFont{}
%\def\InfoPageFont{\small}  % uncomment to decrease font size

% If you are studing in a Czech programme, you also need to provide metadata in Czech:
% (in English programmes, this is not used anywhere)

\def\ThesisTitleCS{Přehledový FMCW radar}
\def\DepartmentCS{Katedra elektromagnetického pole}
\def\DepartmentTypeCS{Katedra}
\def\SupervisorsDepartmentCS{Katedra elektromagnetického pole}
\def\StudyProgrammeCS{Elektronika a komunikace}

\def\ThesisKeywordsCS{%
	FMCW, Přehledový radar, ESP32, MATLAB, Číselné zpracování signálů, 3D tisk, G-code
}

\def\AbstractCS{%
Tato bakalářská práce se zabývá realizací univerzálního přehledového radaru založeného na  FMCW modulu.
Jedná se o technologii, která umožňuje měření vzdálenosti cílů s nízkou spotřebou energie.
Pro dosažení tohoto cíle byli nejprve analyzovány schopnosti modulu.
Následně byla navržena a sestavena dvouosá rotační platforma, která umožňuje mechanické natáčení paprsku.
Platforma je řízena ESP32C6 mikrokontrolérem a ovládána příkazy podobné G-kódu.
Pro integraci platformy s radarovým modulem a zpracování nasbíraných dat byl použit MATLAB.
Přestože radaru neumožňuje sledování pohyblivých cílů, systém je schopen monitorovat širokou škálu prostředí.
Ať už se jedná o aplikace na krátkou vzdálenost vyžadující vysoké rozlišení, nebo monitoring prostředí až do vzdálenosti 300 metrů.
Pro tento účel je proces zpracování dat navržen tak, aby byl co nejvíce konfigurovatelný.
}

