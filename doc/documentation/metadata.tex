%%% Please fill in basic information on your thesis, which will be automatically
%%% inserted at the right places. You need to replace ... by real data.

% Type of your thesis:
%	"bc" for Bachelor's
%	"mgr" for Master's
%	"phd" for PhD
%	"rig" for rigorosum
% "sem" for semestral
\def\ThesisType{bc}
\newcommand{\sidar}{SiRad Easy\textsuperscript{\copyright} }


\def\ThesisTitle{Surveillance FMCW Radar}

\def\ThesisTitleShort{Surveillance FMCW Radar}

\def\ThesisAuthor{Havránek Kryštof}

\def\MontSubmitted{May}

\def\YearSubmitted{2024}

\def\Institution{Czech Technical University in Prague}

\def\Faculty{Faculty of Electrical Engineering}

\def\DepartmentType{Department}

\def\Department{Department of Electromagnetic Field}

\def\Supervisor{Ing. Viktor Adler, Ph.D}

\def\SupervisorsDepartment{Department of Electromagnetic Field}

\def\StudyProgramme{Elektronika a komunikace}

\def\Dedication{%
	I would like to express my gratitude to my supervisor, Ing. Viktor Adler, Ph.D., for his guidance and support throughout the process of writing this thesis.
	Their expertise greatly contributed with me understanding this field and had a significant impact on the final outcome of my work.
}

\def\Abstract{%
This thesis presents the development of a surveillance radar system utilizing the SiRad Easy FMCW module.
FMCW (Frequency-Modulated Continuous Wave) technology enables precise distance measurement of targets while maintaining low power consumption.
Initially, the module’s capabilities are analyzed to identify its operational limits.
A custom two-axis rotary platform is then designed and constructed to enable mechanical beam steering, achieving three-dimensional spatial monitoring.
The platform is controlled by an ESP32C6 microcontroller, implementing a G-code-like command interface for convenient operation.
MATLAB is employed to integrate the radar module with the platform to process the acquired data.
While the system’s performance limitations preclude real-time tracking of moving targets, it demonstrates ability to monitor diverse static environments, ranging from high-resolution close-range applications to long-range detection of targets up to 300 meters.
To accommodate these use cases, the processing pipeline is designed with flexibility in mind, enabled by great degree of configurability.
}

% 3 to 5 keywords (recommended) separated by \sep
% Keywords are useful for indexing and searching for the theses by topic.
\def\ThesisKeywords{%
	FMCW, Surveillance Radar, ESP32, MATLAB, Digital Signal Processing, 3D Printing, G-code
}

% If any of your metadata strings contains TeX macros, you need to provide
% a plain-text version for use in XMP metadata embedded in the output PDF file.
% If you are not sure, check the generated thesis.xmpdata file.
\def\AuthorXMP{\ThesisAuthor}
\def\TitleXMP{\ThesisTitle}
\def\KeywordsXMP{\ThesisKeywords}
\def\AbstractXMP{\Abstract}

% If your abstracts are long and do not fit in the infopage, you can make the
% fonts a bit smaller by this setting. (Also, you should try to compress your abstract more.)
\def\InfoPageFont{}
%\def\InfoPageFont{\small}  % uncomment to decrease font size

% If you are studing in a Czech programme, you also need to provide metadata in Czech:
% (in English programmes, this is not used anywhere)

\def\ThesisTitleCS{Přehledový FMCW radar}
\def\DepartmentCS{Katedra elektromagnetického pole}
\def\DepartmentTypeCS{Katedra}
\def\SupervisorsDepartmentCS{Katedra elektromagnetického pole}
\def\StudyProgrammeCS{Elektronika a komunikace}

\def\ThesisKeywordsCS{%
	FMCW, Přehledový radar, ESP32, MATLAB, Číselné zpracování signálů, 3D tisk, G-code
}

\def\AbstractCS{%
Práce se zabývá vývojem přehledového radarového systému využívajícího modul SiRad Easy FMCW.
Technologie FMCW (Frequency-Modulated Continuous Wave) umožňuje přesné měření vzdálenosti cílů při zachování nízké spotřeby energie.
Nejprve jsou analyzovány možnosti modulu za účelem stanovení jeho provozních limitů.
Následně je navržena a sestavena dvouosá rotační platforma pro natáčení radaru, což umožňuje detekci cílů v prostoru.
Platforma je řízena mikrokontrolérem ESP32C6 a ovládána příkazy podobným G-kódu.
Jazyk MATLAB je použit pro integraci radarového modulu s platformou a zpracování získaných dat.
Přestože parametry použitého radarového modulu znemožňují sledování pohyblivých cílů, systém vykazuje schopnost monitorovat různorodé statické prostředí -- od aplikací s vysokým rozlišením na krátkou vzdálenost až po detekci cílů na vzdálenost až 300 metrů.
Proces zpracování dat je navržen s důrazem na flexibilitu, kdy jeho části mají nastavitelné parametry, či se dají vypnout.
}

