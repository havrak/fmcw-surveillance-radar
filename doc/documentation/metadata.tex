%%% Please fill in basic information on your thesis, which will be automatically
%%% inserted at the right places. You need to replace ... by real data.

% Type of your thesis:
%	"bc" for Bachelor's
%	"mgr" for Master's
%	"phd" for PhD
%	"rig" for rigorosum
% "sem" for semestral
\def\ThesisType{bc}


\def\ThesisTitle{Thesis title}

\def\ThesisTitleShort{Surveillance FMCW radar}

\def\ThesisAuthor{Havránek Kryštof}

\def\MontSubmitted{MONTH}

\def\YearSubmitted{YEAR}

\def\Institution{Czech Technical University in Prague}

\def\Faculty{Faculty of Electrical Engineering}

\def\DepartmentType{Department}
\def\Department{Department of Electromagnetic Field}

\def\Supervisor{Ing. Viktor Adler, Ph.D}

\def\SupervisorsDepartment{department}

\def\StudyProgramme{Elektronika a komunikace}

\def\Dedication{%
Dedication.
}

\def\Abstract{%
\xxx{Use the most precise, shortest sentences that state what problem the
thesis addresses, how it is approached, pinpoint the exact result achieved, and
describe the applications and significance of the results. Highlight anything
novel that was discovered or improved by the thesis. Maximum length is 200
words, but try to fit into 120. Abstracts are often used for deciding if
a reviewer will be suitable for the thesis; a well-written abstract thus
increases the probability of getting a reviewer who will like the thesis.}
}

% 3 to 5 keywords (recommended) separated by \sep
% Keywords are useful for indexing and searching for the theses by topic.
\def\ThesisKeywords{%
keyword\sep key phrase
}

% If any of your metadata strings contains TeX macros, you need to provide
% a plain-text version for use in XMP metadata embedded in the output PDF file.
% If you are not sure, check the generated thesis.xmpdata file.
\def\AuthorXMP{\ThesisAuthor}
\def\TitleXMP{\ThesisTitle}
\def\KeywordsXMP{\ThesisKeywords}
\def\AbstractXMP{\Abstract}

% If your abstracts are long and do not fit in the infopage, you can make the
% fonts a bit smaller by this setting. (Also, you should try to compress your abstract more.)
\def\InfoPageFont{}
%\def\InfoPageFont{\small}  % uncomment to decrease font size

% If you are studing in a Czech programme, you also need to provide metadata in Czech:
% (in English programmes, this is not used anywhere)

\def\ThesisTitleCS{Název práce česky}
\def\DepartmentCS{Název katedry česky}
\def\DepartmentTypeCS{Katedra}
\def\SupervisorsDepartmentCS{katedra vedoucího}
\def\StudyProgrammeCS{studijní program}

\def\ThesisKeywordsCS{%
klíčová slova\sep klíčové fráze
}

\def\AbstractCS{%
Abstrakt práce přeložte také do češtiny.
}

