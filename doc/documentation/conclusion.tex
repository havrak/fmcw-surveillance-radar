% vim.ft=tex
\chapter*{Conclusion}
\addcontentsline{toc}{chapter}{Conclusion}

The goal of this thesis was to develop a surveillance radar system based on FMCW technology.
This approach enables accurate distance measurements of targets with relatively low power consumption.
Rather than adopting a more conventional MIMO system, a simpler solution was chosen: a single RX and TX antenna, with mechanical beam steering provided by a custom rotary platform.

Using off-the-shelf components and 3D-printed parts, a two-axis rotary platform was designed and constructed.
Despite minor issues with belt tension on the pitch axis, the platform reliably controls the radar’s position in both yaw and pitch, with a standard accuracy of 1.8°, further improvable via microstepping.
Additional features such as automatic homing and rotation limits were implemented for improved usability.
Thanks to its similarity to other G-code-based systems, the platform is readily adaptable to other applications.
Moreover, the platform was designed with capabilities beyond the requirements of this thesis, and during testing it executed commands accurately and without noticeable delay.

The capabilities of the \sidar evaluation board were analyzed and found to be rather limited for use in surveillance radar.
Its low and inconsistent reporting rate (50~Hz) restricts the maximum detectable speed well below one meter per second.
This limitation effectively prevents the tracking of moving targets -- a common use case for surveillance radars -- and thus significantly reduced the scope of this thesis.
The system is therefore mainly suitable for static scenes, providing binary detection of target presence.
Nevertheless, within this limitation, the radar still provides good resolution across a variety of ranges, thanks to the ability to switch between the 24GHz and 122~GHz headers.
Both close range application with high resolution or long range (up to some 300~meters) are possible.

A MATLAB-based control application was developed to integrate both rotary platform management and radar data processing in a single package.
 The data processing pipeline follows standard approaches, employing techniques such as FFT, CFAR, and DBSCAN, and stores results in radar cubes..
Extensive parallelization was used to ensure timely processing, even for computationally intensive tasks like decaying the entire cube.
This enables application to maintain speed, even if costly operations, entailing millions of floating point operations, such as a decaying of the whole cube, are performed.

Due to the generic design requirements, the processing pipeline is highly customizable.
Settings such as the number of FFT points, step count per rotation, and CFAR parameters can all be tailored for specific applications.
However, this flexibility requires users to have a solid understanding of the underlying algorithms and principles.

Only visual output was implemented for this thesis, with both 2D and 3D visualization options available.
The classical 2D Range-Azimuth map is likely the most useful, providing information about a significant portion of the environment, given the radar’s radiation pattern.
In case of 3D visualization only CFAR data are visualized in exchange for a more detailed view of the environment by providing pitch information.
If deployed in a cluttered environment, the DBSCAN algorithm can be used to filter out unwanted noise, but in this thesis, its implementation is mainly illustrative and not thoroughly optimized or correctly configured.

Importantly, the processing pipeline was designed such that, if the radar module were replaced with a faster unit, the system’s capabilities could be extended without significant changes to the codebase.
More complex operations, such as 2D FFT or cube updates, are executed in parallel and are triggered by platform movement rather than each radar update.
Therefore, increasing the number of chirps would not significantly impact performance.

However, if the number of FFT points (for speed or range) were to increase substantially, beyond what was tested (up to about 500~MB cubes), a different cube update strategy would be needed.
 One possibility would be to leverage GPU acceleration for cube operations.
Alternatively, with only a CPU, the cube could be split into smaller chunks, loading only the relevant portions into RAM according to the platform’s position and direction.
However, the decay operation would be nearly impossible to realize quickly in case of large cubes without GPU acceleration.

In conclusion, a basic surveillance radar system was successfully designed and implemented.
It is capable of detecting targets over a wide area, with a basic resolution of 1 degree in yaw and pitch, and range accuracy from centimeters to decimeters.
The processing pipeline offers extensive customization, allowing users to fine-tune the system to their needs.
However, due to the limitations of the radar module, the system is not suitable for applications requiring the tracking of moving targets.
