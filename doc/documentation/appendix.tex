\chapter*{Appendix}

\section*{Appendix A -- G-code glossary}

\setlist[itemize]{nosep, topsep=3pt, partopsep=3pt, leftmargin=*}
\setlist[description]{nosep, topsep=1pt, partopsep=1pt, leftmargin=*}
Following sections acts as glossary of all G-code commands used in the platform control. Mode indepth description of the commands can be found in the source code.

\subsection{Constants}
\begin{itemize}
  \item Parameters defined in Kconfig (require firmware recompilation):
  \item \texttt{STEPPER\_Y\_STEP\_COUNT}, \texttt{STEPPER\_P\_STEP\_COUNT}: Steps per rotation (modifiable via \texttt{M92}, non-persistent)
  \item \texttt{STEPPER\_[Y/P]\_PIN\_DIR}, \texttt{STEPPER\_[Y/P]\_PIN\_STEP}: GPIO pins for direction/step signals
  \item \texttt{STEPPER\_[Y/P]\_PIN\_ENDSTOP}: Endstop detection pins
  \item \texttt{STEPPER\_MAX\_SPEED}: Maximum RPM (hardware limit)
  \item \texttt{STEPPER\_DEFAULT\_SPEED}: Default motion RPM
  \item \texttt{STEPPER\_MIN\_SPINDLE\_TIME}: Minimum step interval of rotation in spindle mode
  \item \texttt{STEPPER\_HAL\_TIMER\_PERIOD/RESOLUTION}: PWM timer period/resolution, affects achievable RPM range
\end{itemize}

\subsection{Features}
\begin{itemize}
  \item \textbf{Units}: Switch between degrees and steps using \texttt{G20}/\texttt{G21} respectively
  \item \textbf{Positioning Modes}:
    \begin{itemize}
      \item Absolute (\texttt{G90}): Automatically normalizes angles to [0, 360°] or [0, STEP\_COUNT]
      \item Relative (\texttt{G91}): Limited to $\pm$32767 steps per command, or respective angle counterpart
      \item in absolute positioning devices moves in the shortest path to the target
    \end{itemize}
  \item \textbf{Spindle}: continous rotation mode
  \item \textbf{Limits}: Limits on rotation can be imposed on both axes
    \begin{itemize}
      \item limits are applied only to one rotation -- it is not possible to restrict to e.g. two full rotations in one direction a nd one in the other
      \item \texttt{low < high}: Valid range [low, high]
      \item \texttt{low > high}: Valid range [low, 360°] $\cup$ [0, high]
      \item Out-of-range targets snap to nearest valid position
    \end{itemize}
  \item \textbf{Programming}: G-code program can be upload to the device and executed
  \item \textbf{Synchronization}: If command is issued to both axes they will wait for each other, of not other axis will be free to execute next command
\end{itemize}
\newpage

\subsection*{Motion Control Commands}

Axes are denoted as \texttt{Y} for yaw and \texttt{P} for pitch. The \texttt{S} parameter is used for speed for both axes, while \texttt{SY} and \texttt{SP} are used for speed of yaw and pitch respectively. In case of spindle \texttt{Y} or \texttt{P} are not followed by step count/angle but instead rotation direction.

\begin{description}
  \item[\texttt{M80}] Enable high-voltage power (enables drivers)
  \item[\texttt{M81}] Disable high-voltage power (disables drivers)
  \item[\texttt{M82}] Emergency stop (clears command queues)
  \item[\texttt{G20}] Set units to degrees
  \item[\texttt{G21}] Set units to steps
  \item[\texttt{G90 [Y/P]}] Set absolute positioning Mode
    \begin{itemize}
      \item  absolute positioning is not available in spindle mode, if you issue a M03 command to stepper in absolute positioning mode, it will be automatically switched to relative positioning and throw an error
      \item if no argument is provided both axis are switched to relative positioning
    \end{itemize}
  \item[\texttt{G91 [Y/P]}] Set relative positioning Mode
  \item[\texttt{G28 [Y/P]}] Auto-home specified axis(es)
  \item[\texttt{G92 [Y/P]}] Set current position as zero
  \item[\texttt{G0 Y<val> P<val> [SY/SP<rpm>]}] Stepper-mode movement
  \item[\texttt{M03 Y/P<+/-> [SY/SP<rpm>]}] Start spindle mode (continuous rotation)
  \item[\texttt{M05 [Y/P]}] Stop spindle mode
  \item[\texttt{M201 LY/HY/LP/HP<angle>}] Set angular limits (degrees/steps)
  \item[\texttt{M202 [Y/P]}] Disable limits
\end{description}

\subsection*{Special Commands}

These commands bypass certain sections and safety checks of the firmware. Use with caution.

\begin{description}
  \item[\texttt{G3 Y/P<steps> [SY/SP<rpm>]}] Bypass scheduler and adds movement directly to the queue
		\begin{itemize}
			\item limits are not and cannot be checked
			\item absolute positioning is not available
			\item all values are interpreted as steps regardless of current unit setting
			\item \textit{Warning}: Use \texttt{M82} before \texttt{G3} to maintain position integrity
		\end{itemize}
  \item[\texttt{W3 T<ms>}] Application-layer delay
		\begin{itemize}
			\item useful to wait for end of homing process
		\end{itemize}
\end{description}

\subsection*{Programming Commands}
\begin{description}
  \item[\texttt{P90 <id>}] Start programming mode (overwrites existing program if any)
  \item[\texttt{P91}] Transition from header to main program body
  \item[\texttt{P92}] Finalize program (discarded if loops unclosed)
  \item[\texttt{P21 I<iterations>}] For-loop declaration
  \item[\texttt{P22}] End loop block
  \item[\texttt{P29}] Infinite loop marker (header only) - main body will be executed infinitely
  \item[\texttt{W0 Y/P<sec>}] Wait in seconds
  \item[\texttt{W1 Y/P<ms>}] Wait in milliseconds
\end{description}


\subsection*{Uplink Protocol}

\begin{itemize}
  \item \texttt{!P <timestamp>, <yaw>, <pitch>}: Position update (20ms interval)
  \item \texttt{!R OK}: Command acknowledgment
  \item \texttt{!R ERR <code>}: Command error
\end{itemize}

\begin{tabular}{ll}
  \hline
  \texttt{Code} & Description                          \\
  \hline
  1             & Malformed command syntax             \\
  2             & Invalid arguments                    \\
  3             & Queue lock failure                   \\
  4             & Unsupported command                  \\
  5             & System busy (homing/program running) \\
  6             & Runtime exception                    \\
  7             & Unclosed loop in program             \\
  8             & Invalid context                      \\
  \hline
\end{tabular}


\subsection*{Example Program}
\begin{tabular}{lll}
  \hline
  \texttt{Command}       & Mode   & Purpose                   \\
  \hline
  \texttt{P90 prog}      & Header & Initialize program ID=1   \\
  \texttt{G91}           & Header & Set relative positioning  \\
  \texttt{P29}           & Header & Enable infinite looping   \\
  \texttt{M03 SY6 Y+}    & Body   & Start Yaw spindle (6 RPM) \\
  \texttt{G0 SP30 P100}  & Body   & Pitch movement            \\
  \texttt{G0 SP30 P-100} & Body   & Return pitch              \\
  \texttt{P92}           & Body   & Finalize program          \\
  \hline
\end{tabular}
