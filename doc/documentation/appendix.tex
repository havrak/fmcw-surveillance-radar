\chapter*{Appendix}

\section*{Appendix A -- G-code glossary}

\setlist[itemize]{nosep, topsep=3pt, partopsep=3pt, leftmargin=*}
\setlist[description]{nosep, topsep=1pt, partopsep=1pt, leftmargin=*}

Following sections act as a glossary of all G-code commands used in the platform control. More in-depth descriptions of the commands can be found in the source code documentation.

\subsection*{Constants}

\begin{itemize}
  \item Parameters defined in Kconfig (require firmware recompilation):
  \item \texttt{STEPPER\_Y\_STEP\_COUNT}, \texttt{STEPPER\_P\_STEP\_COUNT}: Steps per rotation (modifiable via \texttt{M92}, non-persistent)
  \item \texttt{STEPPER\_[Y/P]\_PIN\_DIR}, \texttt{STEPPER\_[Y/P]\_PIN\_STEP}: GPIO pins for direction/step signals
  \item \texttt{STEPPER\_[Y/P]\_PIN\_ENDSTOP}: Endstop detection pins
  \item \texttt{STEPPER\_MAX\_SPEED}: Maximum RPM (hardware limit)
  \item \texttt{STEPPER\_DEFAULT\_SPEED}: Default motion RPM
  \item \texttt{STEPPER\_MIN\_SPINDLE\_TIME}: Minimum step interval of rotation in spindle mode
  \item \texttt{STEPPER\_HAL\_TIMER\_PERIOD/RESOLUTION}: PWM timer period/resolution, affects achievable RPM range
\end{itemize}

\subsection*{Features}

\begin{itemize}
  \item \textbf{Units}: Switch between degrees and steps using \texttt{G20}/\texttt{G21} respectively
  \item \textbf{Positioning Modes}:
    \begin{itemize}
      \item Absolute (\texttt{G90}): Automatically normalizes angles to [0, 360°] or [0, STEP\_COUNT]
      \item Relative (\texttt{G91}): Limited to $\pm$32767 steps per command, or respective angle counterpart
      \item in absolute positioning devices moves in the shortest path to the target
    \end{itemize}
  \item \textbf{Spindle}: Continuous rotation mode
  \item \textbf{Limits}: Limits on rotation can be imposed on both axes
    \begin{itemize}
      \item Limits are applied only to one rotation—it is not possible to restrict to, e.g., two full rotations in one direction and one in the other
      \item \texttt{low < high}: Valid range [low, high]
      \item \texttt{low > high}: Valid range [low, 360°] $\cup$ [0, high]
      \item Out-of-range targets snap to nearest valid position
    \end{itemize}
  \item \textbf{Programming}: G-code program can be uploaded to the device and executed
  \item \textbf{Synchronization}: If a command is issued to both axes, they will wait for each other; otherwise, the other axis will be free to execute the next command
\end{itemize}
\newpage

\subsection*{Motion Control Commands}

Axes are denoted as \texttt{Y} for yaw and \texttt{P} for pitch. The \texttt{S} parameter is used for speed for both axes, while \texttt{SY} and \texttt{SP} are used for the speed of yaw and pitch respectively. In the case of spindle, \texttt{Y} or \texttt{P} are not followed by step count/angle but instead by rotation direction.

\begin{description}
  \item[\texttt{M80}] Enable high-voltage power (enables drivers)
  \item[\texttt{M81}] Disable high-voltage power (disables drivers)
  \item[\texttt{M82}] Emergency stop (clears command queues)
  \item[\texttt{G20}] Set units to degrees
  \item[\texttt{G21}] Set units to steps
  \item[\texttt{G90 [Y/P]}] Set absolute positioning Mode
    \begin{itemize}
      \item Absolute positioning is not available in spindle mode; if you issue an M03 command to a stepper in absolute positioning mode, it will be automatically switched to relative positioning and throw an error
      \item If no argument is provided, both axes are switched to relative positioning
    \end{itemize}
  \item[\texttt{G91 [Y/P]}] Set relative positioning Mode
  \item[\texttt{G28 [Y/P]}] Auto-home specified axis(es)
  \item[\texttt{G92 [Y/P]}] Set current position as zero
  \item[\texttt{G0 Y<val> P<val> [SY/SP<rpm>]}] Stepper-mode movement
  \item[\texttt{M03 Y/P<+/-> [SY/SP<rpm>]}] Start spindle mode (continuous rotation)
  \item[\texttt{M05 [Y/P]}] Stop spindle mode
  \item[\texttt{M201 LY/HY/LP/HP<angle>}] Set angular limits (degrees/steps)
  \item[\texttt{M202 [Y/P]}] Disable limits
\end{description}

\subsection*{Special Commands}

These commands bypass certain sections and safety checks of the firmware. Use with caution.

\begin{description}
  \item[\texttt{G3 Y/P<steps> [SY/SP<rpm>]}] Bypass scheduler and adds movement directly to the queue
    \begin{itemize}
      \item Limits are not and cannot be checked
      \item Absolute positioning is not available
      \item All values are interpreted as steps regardless of current unit setting
      \item \textit{Warning}: Use \texttt{M82} before \texttt{G3} to maintain position integrity
    \end{itemize}
  \item[\texttt{W3 T<ms>}] Application-layer delay
    \begin{itemize}
      \item Useful to wait for end of homing process
    \end{itemize}
\end{description}

\subsection*{Programming Commands}
\begin{description}
  \item[\texttt{P90 <id>}] Start programming mode (overwrites existing program if any)
  \item[\texttt{P91}] Transition from header to main program body
  \item[\texttt{P92}] Finalize program (discarded if loops unclosed)
  \item[\texttt{P21 I<iterations>}] For-loop declaration
  \item[\texttt{P22}] End loop block
  \item[\texttt{P29}] Infinite loop marker (header only) - main body will be executed infinitely
  \item[\texttt{W0 Y/P<sec>}] Wait in seconds
  \item[\texttt{W1 Y/P<ms>}] Wait in milliseconds
\end{description}


\subsection*{Uplink Protocol}

\begin{itemize}
  \item \texttt{!P <timestamp>, <yaw>, <pitch>}: Position update (20ms interval)
  \item \texttt{!R OK}: Command acknowledgment
  \item \texttt{!R ERR <code>}: Command error
\end{itemize}

\begin{tabular}{ll}
  \hline
  \texttt{Code} & Description                          \\
  \hline
  1             & Malformed command syntax             \\
  2             & Invalid arguments                    \\
  3             & Queue lock failure                   \\
  4             & Unsupported command                  \\
  5             & System busy (homing/program running) \\
  6             & Runtime exception                    \\
  7             & Unclosed loop in program             \\
  8             & Invalid context                      \\
  \hline
\end{tabular}


\subsection*{Example Programs}

Keep in mind that when using the interface from the application commands \texttt{P90}, \texttt{P91}, \texttt{P92} should not be entered. The program will be automatically finalized when user starts upload to the platform.

\vspace{0.3cm}
\noindent
\textbf{Example 1:} Continuos rotation in yaw, up and down in pitch

\begin{tabular}{lll}
  \hline
  \texttt{Command}     & Mode   & Purpose                      \\
  \hline
  \texttt{P90 rotTilt} & Header & Initialize program rotTilt   \\
  \texttt{G91}         & Header & Set relative positioning     \\
  \texttt{G21}         & Header & Set units to steps           \\
  \texttt{G28}         & Header & Start auto home routine      \\
  \texttt{G0 P-50 S6}  & Header & Move pitch 50steps           \\
  \texttt{W3 T5000}    & Header & Wait five second             \\
  \texttt{P92}         & Header & Set current position as home \\
  \texttt{P29}         & Header & Enable infinite looping      \\
  \texttt{M03 SY6 Y+}  & Header & Start Yaw spindle (6 RPM)    \\
  \texttt{P92}         & Header & Finalize header declaration  \\
  \texttt{G0 S5 P40}   & Body   & Pitch movement               \\
  \texttt{G0 S5 P-40}  & Body   & Return pitch                 \\
  \texttt{P92}         & Body   & Finalize program             \\
  \hline
\end{tabular}


\vspace{0.3cm}
\noindent
\textbf{Example 2:} Continuos rotation in yaw, static in pitch


\begin{tabular}{lll}
  \hline
  \texttt{Command}    & Mode   & Purpose                      \\
  \hline
  \texttt{P90 rot}    & Header & Initialize program rot       \\
  \texttt{G91}        & Header & Set relative positioning     \\
  \texttt{G21}        & Header & Set units to steps           \\
  \texttt{G28}        & Header & Start auto home routine      \\
  \texttt{G0 P-50 S6} & Header & Move pitch 50steps           \\
  \texttt{W3 T5000}   & Header & Wait five second             \\
  \texttt{G92}        & Header & Set current position as home \\
  \texttt{P29}        & Header & Enable infinite looping      \\
  \texttt{M03 SY6 Y+} & Header & Start Yaw spindle (6 RPM)    \\
  \texttt{P91}        & Header & Finalize header declaration  \\
  \texttt{P92}        & Body   & Finalize program             \\
  \hline
\end{tabular}


\vspace{0.3cm}
\noindent
\textbf{Example 3:} Half rotation in yaw, static in pitch


\begin{tabular}{lll}
  \hline
  \texttt{Command}     & Mode   & Purpose                                        \\
  \hline
  \texttt{P90 halfrot} & Header & Initialize program halfrot                     \\
  \texttt{G91}         & Header & Set relative positioning                       \\
  \texttt{G21}         & Header & Set units to steps                             \\
  \texttt{G28}         & Header & Start auto home routine                        \\
  \texttt{G0 P-50 S6}  & Header & Move pitch 50steps                             \\
  \texttt{W3 T5000}    & Header & Wait five second                               \\
  \texttt{G92}         & Header & Set current position as home                   \\
  \texttt{G0 Y-50 S6}  & Header & Shift 50 steps so that movement is around zero \\
  \texttt{P29}         & Header & Enable infinite looping                        \\
  \texttt{P91}         & Header & Finalize header declaration                    \\
  \texttt{G0 Y100 S6}  & Body & Move clockwise                                 \\
  \texttt{G0 Y-100 S6} & Body & Move counterclockwise                          \\
  \texttt{P92}         & Body   & Finalize program                               \\
  \hline
\end{tabular}

\vspace{0.3cm}
\noindent
\textbf{Example 4:} Partial rotation in yaw, static in pitch, absolute mode


\begin{tabular}{lll}
  \hline
  \texttt{Command}    & Mode   & Purpose                       \\
  \hline
  \texttt{P90 parrot} & Header & Initialize program parrot     \\
  \texttt{G91}        & Header & Set relative positioning      \\
  \texttt{G20}        & Header & Set units to degrees          \\
  \texttt{G28}        & Header & Start auto home routine       \\
  \texttt{G0 P-30 S6} & Header & Move pitch 30 degrees         \\
  \texttt{W3 T5000}   & Header & Wait five second              \\
  \texttt{G92}        & Header & Set current position as home  \\
  \texttt{G90}        & Header & Set absolute positioning mode \\
  \texttt{G0 Y80 S6}  & Header & Move to 80 degrees            \\
  \texttt{P29}        & Header & Enable infinite looping       \\
  \texttt{P91}        & Header & Finalize header declaration   \\
  \texttt{G0 Y280 S6} & Body & Move to 280 degrees           \\
  \texttt{G0 Y80 S6}  & Body & Move to 80 degrees            \\
  \texttt{P92}        & Body   & Finalize program              \\
  \hline
\end{tabular}

\vspace{0.3cm}
\noindent
\textbf{Example 5:} Squares


\begin{tabular}{lll}
  \hline
  \texttt{Command}        & Mode   & Purpose                      \\
  \hline
  \texttt{P90 square}     & Header & Initialize program square    \\
  \texttt{G91}            & Header & Set relative positioning     \\
  \texttt{G20}            & Header & Set units to degrees         \\
  \texttt{G28}            & Header & Start auto home routine      \\
  \texttt{G0 P-30 S6}     & Header & Move pitch 30 degrees        \\
  \texttt{W3 T5000}       & Header & Wait five second             \\
  \texttt{G92}            & Header & Set current position as home \\
  \texttt{P29}            & Header & Enable infinite looping      \\
  \texttt{P91}            & Header & Finalize header declaration  \\
  \texttt{G0 Y20 P0 S6}   & Body & Move yaw, fix pitch          \\
  \texttt{G0 Y0 P20 S10}  & Body & Move pitch, fix yaw          \\
  \texttt{G0 Y20 P0 S6}   & Body & Move yaw, fix pitch          \\
  \texttt{G0 Y0 P-20 S10} & Body & Move pitch, fix yaw          \\
  \texttt{P92}            & Body   & Finalize program             \\
  \hline
\end{tabular}
